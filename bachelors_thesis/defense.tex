\documentclass{beamer}
\usepackage[utf8]{inputenc}
\usepackage{graphicx} % Required for inserting images
\usepackage{blindtext}

\usepackage{mathtools} % Mathe
\usepackage{float}
\usepackage{amsmath, amsthm, amssymb} % Mathe
\usepackage{tikz} % Bilder
\usepackage[english]{babel}
\usepackage{babelbib}
\usepackage{url}
\usepackage{tabularx}

\usepackage{cmap}
\usepackage[T1]{fontenc}
\usepackage{pythonhighlight}
\usepackage{listings}
%\usepackage[hidelinks]{hyperref}

\title{Bachelors defense}
\author{Edgar Marquardt}
\date{\today}

\begin{document}

\maketitle

\begin{frame}{Contents}
    \tableofcontents
\end{frame}

\begin{frame}{Plan}
    \begin{columns}
        \begin{column}{0.5\textwidth}
        My idea
        \begin{itemize}
            \item Motivation LWFA
            \item Flying focus as solution?
            \item programs
            \item Lasy lasers in PIConGPU
            \item flying focus laser simulations
            \item Summary
            \item What now?
        \end{itemize}
        \end{column}
        \begin{column}{0.5\textwidth}
        With Jessicas help
        \begin{itemize}
            \item Why? DLWFA
            \item Flying focus in PIConGPU
            
            \item Lasy + implementation
            \item Flying focus doesnt work - why?
            \begin{itemize}
                \item tests
                \item tests
            \end{itemize}
            \item Conclusion
            \begin{itemize}
                \item why doesnt it work
                \item Now Lasy lasers available in PIConGPU
                \item back to LWFA
            \end{itemize}
        \end{itemize}
        \end{column}
    \end{columns}
\end{frame}

\section{DLWFA}
\begin{frame}{LWFA \cite{LWFA}}
    \begin{columns}
        \begin{column}{1.15\textwidth}
        \includegraphics[width=1\columnwidth]{Images/LWFA.png}
        \end{column}
    \end{columns}
    
\end{frame}

\section{Flying focus lasers in PIConGPU}
\begin{frame}{Flying focus lasers}
    \includegraphics[width=\textwidth]{Images/palastro_flfoc.png}
    {\scriptsize The flying focus setup. Image taken from Palastro et al \cite{Palastro:2020}.}
    \begin{columns}
        \begin{column}{0.5\textwidth}
        \begin{itemize}
            \item Built from an axiparabola and a radial group delay echelon (RGD)
            \item Axiparabola:\begin{itemize}
            \item Focuses light onto a line
            \item ?
            \end{itemize}
            \item RGD:\begin{itemize}
            \item ?
            \end{itemize}
        \end{itemize}
        \end{column}
        \begin{column}{0.5\textwidth}
        \includegraphics[width=\columnwidth]{Images/axiparabola.png}
        {\scriptsize Axiparabola functionality. Image taken from Smartsev et al \cite{Smartsev:2019}.}
        \end{column}
    \end{columns}
\end{frame}
\begin{frame}{More flying focus stuff?}

\end{frame}
\begin{frame}{Lasy \cite{lasydoc}}
    \begin{columns}
        \begin{column}{0.5\textwidth}
        \begin{itemize}
        \item A python library for simulating Laser pulses in a vacuum
        \item Uses complex envelope of the laser field
        \item angular spectrum propagation
        \end{itemize}
        \vspace*{30pt}
        {\scriptsize Images: Example of a Gaussian pulse being propagated by Lasy. Top: generated at the focus, Bottom: 6 $z_R$ after the focus.}
        \end{column}
        \begin{column}{0.5\textwidth}
        \includegraphics[width=\columnwidth]{Images/lasy_gauss_1.png}
        \includegraphics[width=\columnwidth]{Images/lasy_gauss_2.png}
        
        \end{column}
    \end{columns}
\end{frame}
\begin{frame}{Implementing the flying focus: RGD}
    \begin{columns}
        \begin{column}{0.5\textwidth}
        \begin{itemize}
        \item Implemented from scratch as Lasy optical element
        \item Following the description by Ambat et al \cite{Ambat:2023}
        \item Shapes the pulse temporally without focusing or defocussing
        \end{itemize}
        \vspace*{30pt}
        {\scriptsize Images: A Gaussian pulse after interacting with the RGD. Top: field envelope, Bottom: Test results. even after long distances the shape still holds.}
        \end{column}
        \begin{column}{0.5\textwidth}
        \includegraphics[width=\columnwidth]{Images/lasy_rgd0.png}
        \includegraphics[width=\columnwidth]{Images/lasy_rgd_ts.png}
        \end{column}
    \end{columns}
\end{frame}
\begin{frame}{Implementing the flying focus: RGD}
	\begin{columns}
	\begin{column}{0.8\textwidth}
	\includegraphics[width=\columnwidth]{Images/lasy_rgd_field.png}\\
	{\scriptsize The electric field of the laser after interacting with the RGD.}
	\end{column}
	\end{columns}
\end{frame}
\begin{frame}{Implementing the flying focus: Axiparabola}
    \begin{columns}
        \begin{column}{0.5\textwidth}
        \begin{itemize}
        \item Included in Lasy
        \item Following Smartsev et al \cite{Smartsev:2019}
        \item ?
        \end{itemize}
        \vspace*{30pt}
        {\scriptsize Images: A super-Gaussian laser pulse after reflecting off the axiparabola. Top: in the near field, Bottom: in the far field at the beginning of the focus region.}
        \end{column}
        \begin{column}{0.5\textwidth}
        \includegraphics[width=\columnwidth]{Images/lasy_axiparabola.png}
        \includegraphics[width=\columnwidth]{Images/lasy_axiparabola_focus2.png}
        \end{column}
    \end{columns}
\end{frame}
\begin{frame}{Implementing the flying focus: Axiparabola}
	\begin{columns}
	\begin{column}{0.8\textwidth}
	\includegraphics[width=\columnwidth]{Images/lasy_axiparabola_focus.png}\\
	{\scriptsize The electric field of the laser at the beginning of the focus region of the axiparabola.}
	\end{column}
	\end{columns}
\end{frame}
\begin{frame}{Importing to PIConGPU}
    \begin{columns}
        \begin{column}{0.5\textwidth}
        \begin{itemize}
        \item New module \pyth{full_field}
        \item ...
        \end{itemize}
        \end{column}
        \begin{column}{0.5\textwidth}
        
        \end{column}
    \end{columns}
\end{frame}

\section{Testing the flying focus laser}
\begin{frame}{Test 1}
    \begin{columns}
        \begin{column}{0.5\textwidth}
        
        \end{column}
        \begin{column}{0.5\textwidth}
        
        \end{column}
    \end{columns}
\end{frame}
\begin{frame}{Test 2}
    \begin{columns}
        \begin{column}{0.5\textwidth}
        
        \end{column}
        \begin{column}{0.5\textwidth}
        
        \end{column}
    \end{columns}
\end{frame}

\section{Conclusion and Outlook}
\begin{frame}{Remaining Possible reasons for failure}
    \begin{itemize}
    \item The Axiparabola
    \item The Propagation
    \item The Findings in the other papers
    \end{itemize}
\end{frame}
\begin{frame}{Now Lasy lasers available in PIConGPU}
    
\end{frame}
\begin{frame}{Back to LWFA?}
    
\end{frame}

\section{References}
\begin{frame}[allowframebreaks]{References}
    \bibliographystyle{plain}
    \bibliography{refs}
\end{frame}

\end{document}
