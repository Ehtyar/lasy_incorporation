%
% Presentation template for Beamer LaTeX class with HZDR/HIF styling
%
% (C) 2020--\today Alexander Grahn, HZDR, Institut fuer Ressourcenökologie
%
% Version 20230920
%
\documentclass[
  %%%%%%%%%%%%%%%%%%%%%%%%%%%%%%%%%%%%%%%
  % enable this for SVG / disable for PDF output
%  hypertex,dvisvgm,
  % for SVG, typeset presentation multiple times with
  % dvilualatex talk
  % dvisvgm --zoom=-1 --font-format=woff2 --bbox=papersize --page=1- --linkmark=none talk.dvi
  %%%%%%%%%%%%%%%%%%%%%%%%%%%%%%%%%%%%%%%
  % HIF people may want to uncomment this
%  hif,
  %%%%%%%%%%%%%%%%%%%%%%%%%%%%%%%%%%%%%%%
  %
  % really cool, but only available in SVG output
%  laserspot,
  %
  %uncomment this to suppress the Dresden Concept logo
%  noddclogo,
  %
  %uncomment this if you don't like the navigation buttons (prev/next slide, full-screen)
  nonavigation,
  %
  %uncomment this if you want the title page and/or part page text to be aligned flushleft
%  titlepageflushleft,
%  partpageflushleft,
  %
%  aspectratio=169, % for wide-screen (16:9) projectors enable this
  aspectratio=43, % old 4:3 projectors
  %
  % default font size
  10pt
]{beamer}

%%%%%%%%%%%%%%%%%%%%
% language settings
%%%%%%%%%%%%%%%%%%%%
\usepackage[ngerman,english]{babel} % primary: US-English
%\usepackage[english,ngerman]{babel} % primary: German
%%%%%%%%%%%%%%%%%%%%

% font settings; change as needed
\usepackage[T1]{fontenc}
\usepackage{lmodern}
\usefonttheme[onlymath]{serif}

%%%%%%%%%%%%%%%%%%%%%%%%%%%%%%%%%%%%%%%%%%%%%%%%%%%%%%%%%%%%%%%%
% add your packages here
%%%%%%%%%%%%%%%%%%%%%%%%%%%%%%%%%%%%%%%%%%%%%%%%%%%%%%%%%%%%%%%%
% add directories to be searched for graphics files (in braces {...})
\graphicspath{{img/}} % e. g. \graphicspath{{img/}{myfancyplots/}}

\usepackage{booktabs} % nicely spaced tables

\usepackage{fancyvrb} % \Verb and Verbatim (environment) for verbatim text
\pdfstringdefDisableCommands{\def\Verb+#1+{#1}} % prevent hyperref warning

%\usepackage{mathtools} % advanced formatting of math formulae

\usepackage[normalem]{ulem} % \uline for underlining main author
\ExplSyntaxOn
\pdfstringdefDisableCommands{\def\uline#1{\str_uppercase:f{#1}}} % prevent hyperref warning
\ExplSyntaxOff

\usepackage{tikz}

% packages below are needed for examples in the video and animations part
%\ifpdf\else
%\usepackage{media4svg} % video inclusion in SVG
%\usepackage{menukeys}  % typesetting keyboard buttons
%\usepackage{siunitx}   % SI units correctly formatted, e.g. \qty{1}{\kilogram}
%\usepackage{pst-ode}   % solve systems of ODEs in LaTeX documents
%\usepackage{pgfplots}  % scientific plots
%\pgfplotsset{compat=newest}
%\usepgfplotslibrary{fillbetween}
%\usetikzlibrary{arrows.meta, calc, decorations}
%\usepackage{listofitems} % read lists of items into array-like structures
%\usepackage{animate}   % PDF/SVG animations
%\usepackage{xsavebox}  % efficient saveboxes for LaTeX
%\fi
%%%%%%%%%%%%%%%%%%%%%%%%%%%%%%%%%%%%%%%%%%%%%%%%%%%%%%%%%%%%%%%%

% load the HZDR theme, finally
\makeatletter\def\input@path{{beamerthemehzdr/}}\makeatother
\usetheme{hzdr}

%%%%%%%%%%%%%%%%%%%%%%%%%%%%%%%%%%%%%%%%%%%%%%%%%%%%%%%%%%%%%%%%%%%%%%%%%%%%%%
%    title, author, date, institute, additional logos
%%%%%%%%%%%%%%%%%%%%%%%%%%%%%%%%%%%%%%%%%%%%%%%%%%%%%%%%%%%%%%%%%%%%%%%%%%%%%%

%\title[short version, used in the footline]{long version in the title page}
\title[Flying focus laser pulses for PIConGPU]{Generating and testing flying focus laser pulses with Lasy for PIConGPU simulations}
\subtitle{---\,A Bachelors Defense\,---} % optional

%\author[short version of authors (not used, currently)]{long version in the title page}
\author[E. Marquardt]{Edgar Marquardt}

%\date[short date]{long date}
%
% Examples using `datetime2' package
\date[\today]{{\DTMsetregional[text]\today}}   % today's date
%\DTMsavenoparsedate{piday22}{2022}{03}{14}{0} % a fixed date, e. g. Monday, 14th March '22 (international Pi Day)
%\date[\DTMusedate{piday22}]{{\DTMsetregional[text]\DTMusedate{piday22}}}

\institute[
  %short version of main authors' institute, used in the footline (not used, currently)
  \iflanguage{ngerman}{%
    Institut für Strahlenphysik%
  }{%
    Institute of Radiation Physics%
  }%
]{%
  %long version, used in the title page
  \iflanguage{ngerman}{%
    Institut für Strahlenphysik, HZDR\\und Institut für Kern- und Teilchenphysik, TU Dresden%
  }{%
    Institute of Radiation Physics, HZDR\\and Institute of Nuclear and Particle Physics, TU Dresden%
  }%
}

%%%%%%%%%%%%%%%%%%
% partner logos
%%%%%%%%%%%%%%%%%%

%every invocation of \partnerlogo{...} adds yet another logo to the
%right of already inserted partner logos; logos are automatically scaled
\partnerlogo{\includegraphics{img/TU_Logo_SW.pdf}}
%\partnerlogo{\fboxsep=1em\colorbox{gray!40}{Logo 2}}

%%%%%%%%%%%%%%%%%%%%%%%%%%%%%%%%%%%%%%%%%%%%%%%%%%%%%%%%%%%%%%%%%%%%%%%%%%%%%%

%hzdr colours used for various text highlights; see manual (texdoc beamer),
%for further possibilities
                                                 %colour will be used in:
\setbeamercolor{alerted text}{fg=hzdr-orange-60} %\alert{text to be highlighted},
                                                 %\begin{alertenv} ... \end{alertenv}
                                                 %\begin{alertblock}{block title (highlighted)}
                                                 %  ...
                                                 %\end{alertblock}

\setbeamercolor{example text}{fg=hzdr-blue-60} %\begin{exampleblock}{block title (highlighted)}
                                               %  ...
                                               %\end{exampleblock}
\hypersetup{%
  breaklinks,colorlinks,
  pdfpagemode=FullScreen,
  linkcolor=hzdr-blue,
  anchorcolor=hzdr-blue,
  citecolor=hzdr-blue,
  filecolor=hzdr-blue,
  menucolor=hzdr-blue,
  runcolor=hzdr-blue,
  urlcolor=hzdr-blue
}


\usepackage{pythonhighlight}
\usepackage{listings}
\usepackage{babelbib}
\usepackage{blindtext}

\usepackage{mathtools} % Mathe
\usepackage{float}
\usepackage{amsmath, amsthm, amssymb} % Mathe
%%%%%%%%%%%%%%%%%%%%%%%%%%%%%%%%%%%%%%%%%%%%%%%%%%%%%%%%%%%%%%%%%%%%%%%%%%%%%%
%                            document starts here                            %
%%%%%%%%%%%%%%%%%%%%%%%%%%%%%%%%%%%%%%%%%%%%%%%%%%%%%%%%%%%%%%%%%%%%%%%%%%%%%%
\begin{document}

\frame{\titlepage}

\frame{\tableofcontents}


\section{Dephasingless Laser WakeField Acceleration (DLWFA)}
\begin{frame}{Laser WakeField Acceleration (LWFA) \cite{LWFA}}
    \begin{columns}
        \begin{column}{1\textwidth}
        \includegraphics[width=0.93\columnwidth]{Images/LWFA.png}
        {\scriptsize Electric field and electrons in an LWFA simulation.}
        \end{column}
    \end{columns}
\end{frame}

\begin{frame}{Flying focus lasers -- solving the Problem of Dephasing}{1. TWEAC \cite{LWFAd}}
	\begin{columns}
        \begin{column}{0.6\textwidth}
        \begin{itemize}
        \item Traveling-Wave Electron ACcelerator
        \item Uses two laser pulses with tilted pulse fronts
        \item The tilt controls the velocity of the overlapping region
        \end{itemize}
        \vspace*{30pt}
        {\scriptsize Image: TWEAC setup using two laser pulses. Image taken from Debus \cite{tweac}}
        \end{column}
        \begin{column}{0.4\textwidth}
        \only<1>{\includegraphics[width=0.9\columnwidth]{Images/tweac1.png}}\only<2>{\includegraphics[width=0.9\columnwidth]{Images/tweac2.png}}\only<3>{\includegraphics[width=0.9\columnwidth]{Images/tweac3.png}}
        \end{column}
    \end{columns}
\end{frame}
\begin{frame}{Flying focus lasers -- solving the Problem of Dephasing}{2. Axiparabola laser \cite{Palastro:2020}}
    \includegraphics[width=\textwidth]{Images/palastro_flfoc.png}
    {\scriptsize The flying focus setup. Two optical elements: The Axiparabola (left) and the Radial Group Delay echelon (RGD) (right). Image taken from Palastro et al \cite{Palastro:2020}.}
\end{frame}
\begin{frame}{Flying focus lasers -- solving the Problem of Dephasing}{2. Axiparabola laser \cite{Palastro:2020}}
    \begin{columns}
        \begin{column}{0.5\textwidth}
        \begin{itemize}
            \item Axiparabola \cite{Smartsev:2019}:\begin{itemize}
            \item Near-parabolic mirror
            \item Focuses light onto a line -- the focus region
            \item[$\rightarrow$] Light at radius $r$ is focused at $f(r)=f_0+\delta\left(\frac{r}{R}\right)^2$
            \end{itemize}
            \item Radial Group Delay echelon (RGD) \cite{Ambat:2023}\cite{Palastro:2020}:\begin{itemize}
            \item Stepped concentric mirror rings
            \item Shape follows some function $\tau_D(r)$
            \item[$\rightarrow$] controls the timing of the axiparabola focus
            \end{itemize}
        \end{itemize}
        \end{column}
        \begin{column}{0.5\textwidth}
        \includegraphics[width=\columnwidth]{Images/axiparabola.png}
        {\scriptsize Axiparabola functionality. Image taken from Smartsev et al \cite{Smartsev:2019}.}
        \end{column}
    \end{columns}
\end{frame}


\section{Flying focus lasers in Lasy and PIConGPU}
\begin{frame}{Lasy \cite{lasydoc}}{A python library}
    \begin{columns}
        \begin{column}{0.55\textwidth}
        \begin{itemize}
        \item A python library for simulating Laser pulses in a vacuum
        \item Uses complex envelope of the laser field
        \item Uses angular spectrum propagation
        \item Can use cylindrical coordinates for memory and CPU time efficiency
        \item Offers a range of optical elements
        \end{itemize}
        \vspace*{30pt}
        {\scriptsize Images: Example of a Gaussian pulse being propagated by Lasy. Top: generated at the focus, Bottom: 6 $z_R$ after the focus.}
        \end{column}
        \begin{column}{0.43\textwidth}
        \includegraphics[width=\columnwidth]{Images/lasy_gauss_1.png}
        \includegraphics[width=\columnwidth]{Images/lasy_gauss_2.png}
        \end{column}
    \end{columns}
\end{frame}
\begin{frame}{Implementing the flying focus}{1. The Radial Group delay echelon (RGD)}
    \begin{columns}
        \begin{column}{0.55\textwidth}
        \begin{itemize}
        \item Implemented from scratch as Lasy optical element
        \item Following the description by Ambat et al \cite{Ambat:2023}
        \item Shapes the pulse temporally without focusing or defocussing
        \item Can generate any radially symmetric shape of delay $\tau_D(r)$
        \end{itemize}
        \vspace*{30pt}
        {\scriptsize Images: A Gaussian pulse after interacting with the RGD. Top: field envelope, Bottom: Test results. even after long distances the shape still holds.}
        \end{column}
        \begin{column}{0.45\textwidth}
        \includegraphics[width=0.9\columnwidth]{Images/lasy_rgd0.png}
        \includegraphics[width=\columnwidth]{Images/lasy_rgd_ts.png}
        \end{column}
    \end{columns}
\end{frame}
\begin{frame}{Implementing the flying focus}{1. The Radial Group delay echelon (RGD)}
	\begin{columns}
	\begin{column}{0.7\textwidth}
	\includegraphics[width=\columnwidth]{Images/lasy_rgd_field.png}\\
	{\scriptsize The electric field of the laser after interacting with the RGD.}
	\end{column}
	\end{columns}
\end{frame}
\begin{frame}{Implementing the flying focus}{2. The Axiparabola}
    \begin{columns}
        \begin{column}{0.55\textwidth}
        \begin{itemize}
        \item Included in Lasy
        \item Following Smartsev et al \cite{Smartsev:2019}
        \item Focusing laser pulse in the focus region
        \item Also implemented an axiparabola following Ambat et al \cite{Ambat:2023}\begin{itemize}
            \item[$\rightarrow$] Very small differences
        \end{itemize}
        \end{itemize}
        \vspace*{30pt}
        {\scriptsize Images: A super-Gaussian laser pulse after reflecting off the axiparabola. Top: in the near field, Bottom: in the far field at the beginning of the focus region.}
        \end{column}
        \begin{column}{0.44\textwidth}
        \includegraphics[width=\columnwidth]{Images/lasy_axiparabola.png}
        \includegraphics[width=\columnwidth]{Images/lasy_axiparabola_focus2.png}
        \end{column}
    \end{columns}
\end{frame}
\begin{frame}{Implementing the flying focus}{2. The Axiparabola}
	\begin{columns}
	\begin{column}{0.7\textwidth}
	\includegraphics[width=\columnwidth]{Images/lasy_axiparabola_focus.png}\\
	\end{column}
	\end{columns}
	\begin{center}
	    {\scriptsize The electric field of the laser at the beginning of the focus region of the axiparabola.}
	\end{center}
\end{frame}
\begin{frame}{Importing to PIConGPU}
    \begin{columns}
        \begin{column}{0.55\textwidth}
        \begin{itemize}
        \item New module \pyth{full_field}
        \item Generates full electric field and saves it using \texttt{openPMD-api}
        \item \textsf{incidentField} method \textsf{FromOpenPMDPulse} \cite{fabia} imports the field into PIConGPU
        \end{itemize}
        \vspace*{30pt}
        {\scriptsize Images: Top: Electric field of the complete flying focus laser at the beginning of the focus region, Bottom: That same electric field entering the simulation window of a PIConGPU simulation.}
        \end{column}
        \begin{column}{0.45\textwidth}
        \includegraphics[width=\columnwidth]{Images/lasy_flfoc_102_focus.png}
        \includegraphics[width=\columnwidth]{Images/pic_flfoc_102_start.png}
        \end{column}
    \end{columns}
\end{frame}
\begin{frame}{Testing the method}
    \begin{columns}
        \begin{column}{0.55\textwidth}
        \begin{itemize}
        \item Test with Gaussian laser pulse and parabolic mirror
        \item Some artifacts are visible\begin{itemize}
            \item Problem of the \textsf{incidentField} method
            \item[$\rightarrow$] See \cite{PICartefact}
        \end{itemize}
        \item Test successful
        \end{itemize}
        \vspace*{30pt}
        {\scriptsize Images: Top: Electric field of the Gaussian laser pulse $z_R$ before the focus as imported into a PIConGPU simulation, Bottom: The beam waist $w$ of the pulse over time, compared to theory.}
        \end{column}
        \begin{column}{0.43\textwidth}
        \includegraphics[width=\columnwidth]{Images/pic_parabol_zr1.png}
        \includegraphics[width=\columnwidth]{Images/pic_parabol_ws.png}
        \end{column}
    \end{columns}
\end{frame}

\section{Testing the flying focus lasers}

\begin{frame}{Testing the flying focus laser}{First results}
    \begin{columns}
        \begin{column}{0.55\textwidth}
        \begin{itemize}
            \item Simulations in Lasy and PIConGPU
            \item Input file for PIConGPU saved at the beginning of the focus region
            \item Result: The flying focus effect cannot be seen
            \item[$\rightarrow$] Test failed
        \end{itemize}
        \vspace*{30pt}
        {\scriptsize Images: Comparing measurements from the simulations to the expected evolution. Top: From a Lasy simulation: $t-z/c$, Bottom: From a PIConGPU simulation: $z-t\cdot c$.}
        \end{column}
        \begin{column}{0.42\textwidth}
        \includegraphics[width=\columnwidth]{Images/lasy_flfoc_102_ts.png}
        \includegraphics[width=\columnwidth]{Images/pic_flfoc_102_ts.png}
        \end{column}
    \end{columns}
\end{frame}
\begin{frame}{Testing the flying focus laser}{Axiparabola only}
    \begin{columns}
        \begin{column}{0.55\textwidth}
        \begin{itemize}
            \item With only the axiparabola differences appear already\begin{itemize}
                \item[$\rightarrow$] Here must be the problem...
            \end{itemize}
            \item 
        \end{itemize}
        \vspace*{30pt}
        {\scriptsize Images: Comparing measurements from a Lasy simulation with theoretical values using Ambat et al \cite{Ambat:2023}. Top: Arrival time $t-z/c$, Bottom: Beam waist $w$.}
        \end{column}
        \begin{column}{0.43\textwidth}
        \includegraphics[width=\columnwidth]{Images/lasy_axiparabola_ts.png}
        \includegraphics[width=\columnwidth]{Images/lasy_axiparabola_ws.png}
        \end{column}
    \end{columns}
\end{frame}
%\begin{frame}{Testing the flying focus laser}{More test?}
%    \begin{columns}
%        \begin{column}{0.55\textwidth}
%        \begin{itemize}
%            \item 
%        \end{itemize}
%        \vspace*{30pt}
%        {\scriptsize Images: }
%        \end{column}
%        \begin{column}{0.41\textwidth}
%        
%        \end{column}
%    \end{columns}
%\end{frame}

\section{Conclusion and Outlook}
\begin{frame}{Conclusion}{Remaining Possible reasons for failure}
    \begin{itemize}
    \item The Axiparabola\begin{itemize}
        \item The shape could still be wrong
    \end{itemize}
    \item The Propagation\begin{itemize}
        \item Possibly the Lasy propagation gives incorrect/incomplete results
    \end{itemize}
    \item The Findings in the other papers\begin{itemize}
        \item Maybe the Axiparabola does not work at all
        \item Maybe it works in a different manner
    \end{itemize}
    \item Maybe something else...
    \end{itemize}
\end{frame}
\begin{frame}{Outlook}
    \begin{itemize}
        \item Lasy lasers available in PIConGPU
        \item[$\rightarrow$] LWFA with new laser setups possible
        \item Some problems still need resolving
        \item ?
    \end{itemize}
        
    \definecolor{ow}{RGB}{250, 180, 0}
    \colorbox{black}{\textcolor{ow}{\texttt{(*) There's more to explore here.}}}
\end{frame}


\section{References}
\begin{frame}[allowframebreaks]{References}
    \bibliographystyle{unsrt}
    \bibliography{refs}
\end{frame}

\end{document}
