\documentclass[
 paper=A4,pagesize=automedia,fontsize=12pt,
 BCOR=15mm,DIV=22,
 twoside,headinclude,footinclude=false,
 english,fleqn,             % fleqn = linksbündige Ausrichtung von Formeln
 bibliography=totoc,          % Literaturverz. im Inhaltsverz. eintragen
 listof=totoc,                % Abbildungsverz. im Inhaltsverz. eintragen
 listof=flat,                 % Abbildungsverz. an der längsten Nummer ausrichten
% numbers=noenddot,          % kein Punkt nach Überschriftsnummerierung
 cleardoublepage=empty      % Vakatseiten ohne Paginierung
 chapterprefix=true
]{scrbook}
\setlength\parindent{0em}

% Kodierung, Schrift und Sprache auswählen
\usepackage[utf8]{inputenc}
\usepackage[T1]{fontenc}
\usepackage[english]{babel}
% damit man Text aus dem PDF korrekt rauskopieren kann
\usepackage{cmap}
% Layout: Kopf-/Fußzeilen, anderthalbfacher Zeilenabstand
\usepackage[headsepline=0.5pt, automark]{scrlayer-scrpage} \pagestyle{scrheadings}
                      \clearscrheadfoot
                      \ihead{\thesection\: \headmark}\ohead{\pagemark}
                      \automark[subsection]{section}
                      \automark[subsubsection]{subsection}
                      %\setheadsepline{0.5pt}
\usepackage{setspace} \onehalfspacing
\deffootnote{1em}{1em}{\textsuperscript{\thefootnotemark }}
% Grafiken, Tabellen, Mathematikumgebungen
\usepackage{graphicx,xcolor}
\usepackage{tabularx}
\usepackage{amsmath,amsfonts,amssymb}
% Darstellung von Fließumgebungen
\usepackage{flafter,afterpage}
\usepackage[section]{placeins}
\usepackage[margin=8mm,font=small,labelfont=bf,format=plain]{caption}
\usepackage[margin=8mm,font=small,labelfont=bf,format=plain]{subcaption}

\numberwithin{equation}{section}
\numberwithin{figure}{section}
\numberwithin{table}{section}

%%%%%%%%%%%%%%%%%%%%%%%%%%%%%%%%%%%%%%%%%%%%%%%%%%%%%%%%%%%%%%%%%%%%%%%%%%%%%%%%
% Ab hier ist Platz für eigene Ergänzungen (Pakete, Befehle, etc.)

% Dieses Paket liefert den Blindtext, der als Platzhalter in den Beispieldateien steht.
% Das kannst Du also entfernen, wenn Du den Blindtext nicht mehr brauchst.
\usepackage{blindtext}
\usepackage{lipsum}
\usepackage{pythonhighlight}
